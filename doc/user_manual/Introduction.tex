\section{Introduction}
\label{sec:Introduction}

SR2ML is a software package that contains a set of safety and reliability models
designed to be interfaced with the INL's RAVEN code. These models can be
employed to perform both static and dynamic system risk analyses and to determine
the risk importance of specific elements of the considered system.

Two classes of reliability models have been developed; the first class includes
all classical reliability models (fault trees, event trees, Markov models and
reliability block diagrams) which have been extended to not only deal with
Boolean logic values but also time-dependent values. The second class includes
several component aging and maintenance models. Models of these two classes are designed to
be included in a RAVEN ensemble model to perform a time-dependent system reliability
analysis (dynamic analysis). Similarly, these models can be interfaced with system
analysis codes  to determine the failure time of systems and evaluate the accident progression
(static analysis).

\subsection{Acquiring and Installing SR2ML}
SR2ML is supported on three separate computing platforms: Linux, OSX (Apple Macintosh), and Microsoft
Windows. Currently, SR2ML is downloadable from the SR2ML GitLab repository:
\url{https://hpcgitlab.hpc.inl.gov/RAVEN_PLUGINS/SR2ML.git}. New users should contact SR2ML developers to
get started. This typically involves the following steps:

\begin{itemize}
  \item \textit{Download SR2ML}
    \\ You can download the source code from \url{https://github.com/idaholab/SR2ML.git}.
	\item \textit{Use as a RAVEN Plugin},
    \\RAVEN must first be downloaded from \url{https://github.com/idaholab/raven.git}.
		\\ Detailed instructions are available from \url{https://github.com/idaholab/raven/wiki}.
    To register a plugin with RAVEN and make its components accessible, run the script:
\begin{lstlisting}[language=bash]
raven/scripts/install_plugins.py -s /abs/path/to/SR2ML
\end{lstlisting}
    After the plugin registration, follow the installation instructions at
    \url{https://github.com/idaholab/raven/wiki/installationMain} to install the
    required dependencies.
\end{itemize}

\subsection{Accessing SR2ML from within RAVEN}
The SR2ML can be accessed as a special subtype of the External model.
The syntax is given in Listing \ref{lst:SR2MLfromRAVEN}.

\begin{lstlisting}[style=XML,morekeywords={anAttribute},caption=Call SR2ML.FTModel from RAVEN input., label=lst:SR2MLfromRAVEN]
<ExternalModel name="FaultTree" subType="SR2ML.FTModel">
  <variables> Input and output variables needed by SR2ML </variables>
  ...
</ExternalModel>
\end{lstlisting}

\subsection{User Manual Formats}
In this manual, we employ the following formats to highlight certain parts with
particular meanings (i.e., input structure, examples, and terminal commands):

\begin{itemize}
\item \textbf{\textit{Python Coding:}}
\begin{lstlisting}[language=python]
class AClass():
  def aMethodImplementation(self):
    pass
\end{lstlisting}
\item \textbf{\textit{SR2ML XML input example:}}
\begin{lstlisting}[style=XML,morekeywords={anAttribute}]
<MainXMLBlock>
  ...
  <aXMLnode anAttribute='aValue'>
     <aSubNode>body</aSubNode>
  </aXMLnode>
  <!-- This is  commented block -->
  ...
</MainXMLBlock>
\end{lstlisting}
\item \textbf{\textit{Bash Commands:}}
\begin{lstlisting}[language=bash]
cd path/to/SR2ML/
cd ../../
\end{lstlisting}
\end{itemize}

\subsection{Capabilities of SR2ML}
This document provides a detailed description of the SR2ML plugin for the RAVEN code~\cite{RAVEN,RAVENtheoryMan}.
The features included in this plugin are:
\begin{itemize}
	\item Event Tree (ET) Model (see Section~\ref{sec:ETModel})
	\item Fault Tree (FT) Model (see Section~\ref{sec:FTModel})
	\item Markov Model (see Section~\ref{sec:MarkovModel})
	\item Reliability Block Diagram (RBD) Model (see Section~\ref{sec:RBDmodel})
	\item Data Classifier (see Section~\ref{sec:dataClassifier})
	\item ET Data Importer (see Section~\ref{sec:ETdataImporter})
	\item FT Data Importer (see Section~\ref{sec:FTdataImporter})
	\item Reliability Models (see Section~\ref{sec:ReliabilityModels})
  \item Maintenance Models (see Section~\ref{sec:MaintenanceModels})
\end{itemize}
