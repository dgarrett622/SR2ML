\section{MCSSolver}
\label{sec:MCSSolver}

This model is designed to read from file a list of Minimal Cut Sets (MCSs) and to import such Boolean logic structure as a RAVEN model.
Provided the sampled values of Basic Events (BEs) probabilities, the MCSSolver determines the probability of Top Event (TE), i.e., the union of the MCSs.
The list of MCS must be provided through a CSV file with the following format:

\begin{table}
  \begin{center}
    \caption{MCS file format.}
    \label{tab:table1}
    \begin{tabular}{c|c|c} 
      \textbf{ID} & \textbf{Prob} & \textbf{MCS}\\
      \hline
      1, & 0.01, & BE1\\
      2, & 0.02, & BE3\\
      3, & 0.03, & BE2,BE4\\
    \end{tabular}
  \end{center}
\end{table}

In this example:
\begin{itemize}
  \item three MCSs are defined: MCS1 = BE1, MCS2 = BE3 and MCS3 = BE2 and BE4 
  \item four BEs are defined: BE1, BE2, BE3 and BE4
  \item probability of TE, i.e. P(TE), is equal to: $P(TE) = P(MCS1 \cup MCS2 \cup MCS3)$
\end{itemize}

Note that the MCSSolver considers only the list of MCSs and it discards the rest of data contained in the csv file.

All the specifications of the MCSSolver model are given in the \xmlNode{ExternalModel} block. 
Inside the \xmlNode{ExternalModel} block, the XML nodes that belong to this models are:
\begin{itemize}
  \item  \xmlNode{variables}, \xmlDesc{string, required parameter}, a list containing the names of both the input and output variables of the model
  \item  \xmlNode{solverOrder},\xmlDesc{integer, required parameter}, solver order for $P(TE)$: it specifies the maximum calculation envelope 
                                                                      for $P(TE)$, i.e., the maximum number of MCSs to be considered when evaluating the 
                                                                      probability of their union
  \item  \xmlNode{topEventID},\xmlDesc{string, required parameter}, the name of the alias variable for the Top Event
  \item  \xmlNode{map},\xmlDesc{string, required parameter}, the name ID of the ET branching variable
	  \begin{itemize}
	    \item \xmlAttr{var}, \xmlDesc{required string attribute}, the ALIAS name ID of the basic event
	  \end{itemize}
\end{itemize}

An example of RAVEN input file is the following:

\begin{lstlisting}[style=XML,morekeywords={anAttribute},caption=MCSSolver model input example., label=lst:MCSSolver_InputExample]
  <Models> 
    ...
    <Models>
      <ExternalModel name="MCSmodel" subType="PRAplugin.MCSSolver">
        <variables>statusBE1,statusBE2,statusBE3,statusBE4,TOP</variables>
        <solverOrder>3</solverOrder>
        <topEventID>TOP</topEventID>>
        <map var='statusBE1'>BE1</map>
        <map var='statusBE2'>BE2</map>
        <map var='statusBE3'>BE3</map>
        <map var='statusBE4'>BE4</map>
      </ExternalModel>
    </Models>
    ...
  </Models>
\end{lstlisting}

If $solverOrder=1$ then: $P(TE) = P(MCS1)+P(MCS2)+P(MCS3)$.  
If $solverOrder=2$ then: $P(TE) = P(MCS1)+P(MCS2)+P(MCS3) - P(MCS1 MCS2) - P(MCS1 MCS3) - P(MCS2 MCS3)$.  
If $solverOrder=3$ then: $P(TE) = P(MCS1)+P(MCS2)+P(MCS3) - P(MCS1 MCS2) - P(MCS1 MCS3) - P(MCS2 MCS3) + P(MCS1 MCS2 MCS3)$

\subsection{MCSSolver model reference tests}
The following is the provided analytic test:
\begin{itemize}
	\item test\_MCSSolver.xml
\end{itemize}



