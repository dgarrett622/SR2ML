\section{MCSSolver}
\label{sec:MCSSolver}

This model is designed to read from file a list of Minimal Cut Sets (MCSs) and to import such Boolean logic structure as a RAVEN model.
Provided the sampled values of Basic Events (BEs) probabilities, the MCSSolver determines the probability of Top Event (TE), i.e., the union of the MCSs.
The list of MCS must be provided through a CSV file with the following format:

\begin{table}
  \begin{center}
    \caption{MCS file format.}
    \label{tab:table1}
    \begin{tabular}{c|c|c} 
      \textbf{ID} & \textbf{Prob} & \textbf{MCS}\\
      \hline
      1, & 0.01, & BE1\\
      2, & 0.02, & BE3\\
      3, & 0.03, & BE2,BE4\\
    \end{tabular}
  \end{center}
\end{table}

In this example:
\begin{itemize}
  \item three MCSs are defined: MCS1 = BE1, MCS2 = BE3 and MCS3 = BE2 and BE4 
  \item four BEs are defined: BE1, BE2, BE3 and BE4
  \item probability of TE, i.e. P(TE), is equal to: $P(TE) = P(MCS1 \cup MCS2 \cup MCS3)$
\end{itemize}

Note that the MCSSolver considers only the list of MCSs and it discards the rest of data contained in the csv file.

All the specifications of the MCSSolver model are given in the \xmlNode{ExternalModel} block. 
Inside the \xmlNode{ExternalModel} block, the XML nodes that belong to this models are:
\begin{itemize}
  \item  \xmlNode{variables}, \xmlDesc{string, required parameter}, a list containing the names of both the input and output variables of the model
  \item  \xmlNode{solverOrder},\xmlDesc{integer, required parameter}, solver order for $P(TE)$: it specifies the maximum calculation envelope 
                                                                      for $P(TE)$, i.e., the maximum number of MCSs to be considered when evaluating the 
                                                                      probability of their union
  \item  \xmlNode{topEventID},\xmlDesc{string, required parameter}, the name of the alias variable for the Top Event
  \item  \xmlNode{map},\xmlDesc{string, required parameter}, the name ID of the ET branching variable
	  \begin{itemize}
	    \item \xmlAttr{var}, \xmlDesc{required string attribute}, the ALIAS name ID of the basic event
	  \end{itemize}
\end{itemize}

An example of RAVEN input file is the following:

\begin{lstlisting}[style=XML,morekeywords={anAttribute},caption=MCSSolver model input example., label=lst:MCSSolver_InputExample]
  <Models> 
    ...
    <Models>
      <ExternalModel name="MCSmodel" subType="SR2ML.MCSSolver">
        <variables>statusBE1,statusBE2,statusBE3,statusBE4,TOP</variables>
        <solverOrder>3</solverOrder>
        <topEventID>TOP</topEventID>>
        <map var='pBE1'>BE1</map>
        <map var='pBE2'>BE2</map>
        <map var='pBE3'>BE3</map>
        <map var='pBE4'>BE4</map>
      </ExternalModel>
    </Models>
    ...
  </Models>
\end{lstlisting}

In this case, the MCSs are written in terms of the variables BE1, BE2, BE3, and BE4.
The values of the variables pBE1, pBE2, pBE3, and pBE4 (i.e., the probability values associated to each basic event) are 
generated outside the MCSSolver model (e.g., by a sampler) and passed to the MCSSolver model in order to calculate the 
probability value of the top event (e.g., the variable TOP in the example above).
The \xmlNode{map} blocks allow the user to link the sampled probability value to the correspoding basic event.

If $solverOrder=1$ then: $P(TE) = P(MCS1)+P(MCS2)+P(MCS3)$.  
If $solverOrder=2$ then: $P(TE) = P(MCS1)+P(MCS2)+P(MCS3) - P(MCS1 MCS2) - P(MCS1 MCS3) - P(MCS2 MCS3)$.  
If $solverOrder=3$ then: $P(TE) = P(MCS1)+P(MCS2)+P(MCS3) - P(MCS1 MCS2) - P(MCS1 MCS3) - P(MCS2 MCS3) + P(MCS1 MCS2 MCS3)$

\subsection{Time dependent calculation}

The MCSSolver can also perform time dependent calculation by providing in the MultiRun step a dataObject which
contains the logic status of the Basic Events.
This dataObject contains the logical status of the the Basic Event: 0 (Basic event set to False: probability=p(BE)) 
or 1 (Basic event set to True: probability=1.0).
The format of the dataObject can be:
\begin{itemize}
  \item HistorySet: it contains the temporal profile of each basic event (e.g., BE1, BE2, BE3, and BE4) as a time 
                    series (see test\_MCSSolver\_TD.xml). In this case it is needed to specify the ID of the time variable 
                    contained in the HistorySet in the \xmlNode{timeID} node.
  
   \begin{lstlisting}[style=XML,morekeywords={anAttribute},caption=Time dependent (from HistorySet) MCSSolver model input example., label=lst:MCSSolver_InputExample]
     <Models>
      <ExternalModel name="MCSmodel" subType="SR2ML.MCSSolver">
        <variables>statusBE1,statusBE2,statusBE3,statusBE4,TOP,time</variables>
        <solverOrder>3</solverOrder>
        <topEventID>TOP</topEventID>
        <timeID>time</timeID>
        <map var='statusBE1'>BE1</map>
        <map var='statusBE2'>BE2</map>
        <map var='statusBE3'>BE3</map>
        <map var='statusBE4'>BE4</map>
      </ExternalModel>
    </Models>
  \end{lstlisting}
  
  \item PointSet: it contains the interval time (initial and final time) under which each basic event is set to 
                  True (see test\_MCSSolver\_TD\_fromPS.xml). As an example, the PointSet contains the IDs of the basic events
                  in the and the initial and final time as shown below:
                  
   \begin{lstlisting}[style=XML,morekeywords={anAttribute},caption= Example of PointSet for time dependent MCSSolver calculation., label=lst:MCSSolver_InputExample]
    <PointSet name="maintenanceSchedule_PointSet">
      <Input>BE</Input>
      <Output>tIn,tFin</Output>
    </PointSet>
    \end{lstlisting}
    
                  In this case, the MCSSolver requires the specification of the DataObject variables that contain the list of basic events
                  (\xmlNode{BE\_ID} node) and the initial and final time values (\xmlNode{tInitial} and \xmlNode{tEnd} nodes).
                  
   \begin{lstlisting}[style=XML,morekeywords={anAttribute},caption=Time dependent (from PointSet) MCSSolver model input example., label=lst:MCSSolver_InputExample]
    <Models>
      <ExternalModel name="MCSmodel" subType="SR2ML.MCSSolver">
        <variables>pBE1,pBE2,pBE3,pBE4,TOP,time,BE1,BE2,BE3,BE4</variables>
        <solverOrder>3</solverOrder>
        <BE_ID>BE</BE_ID>
        <tInitial>tIn</tInitial>
        <tEnd>tFin</tEnd>
        <topEventID>TOP</topEventID>
        <timeID>time</timeID>
        <map var='pBE1'>BE1</map>
        <map var='pBE2'>BE2</map>
        <map var='pBE3'>BE3</map>
        <map var='pBE4'>BE4</map>
      </ExternalModel>
    </Models>
  \end{lstlisting}
\end{itemize}

\subsection{MCSSolver model reference tests}
The following is the provided analytic test:
\begin{itemize}
  \item test\_MCSSolver.xml
  \item test\_MCSSolver\_TD.xml
  \item test\_MCSSolver\_TD\_fromPS.xml
\end{itemize}




