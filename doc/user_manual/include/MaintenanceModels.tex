\section{Maintenance Models}
\label{sec:MaintenanceModels}


\newcommand{\aliasRequiredParameterDescription}[1]
{
	\xmlNode{#1}, \xmlDesc{string or float, required parameter}. See above definition.
	If a string was provided, the reliability model would treat it as an input variable that came
	from other entity of RAVEN. In this case, the variable must be listed in the sub-node
	\xmlNode{variables} under \xmlNode{ExternalModel}.
}
\newcommand{\aliasOptionalParameterDescription}[1]
{
	\xmlNode{#1}, \xmlDesc{string or float, optional parameter}. See above definition.
	If a string was provided, the reliability model would treat it as an input variable that came
	from other entity of RAVEN. In this case, the variable must be listed in the sub-node
	\xmlNode{variables} under \xmlNode{ExternalModel}.
	\default{1.}
}

\textbf{Maintenance Models} are models designed to model maintenance and testing from a reliability perspective. 
These models are designed to optimize preventive maintenance at the system level. 

Two classes of models are here considered:
\begin{itemize}
	\item Operating, i.e. model \xmlAttr{type} is \xmlString{Operating}
	\item Standby, i.e. model \xmlAttr{type} is \xmlString{Standby}
\end{itemize}

The specifications of these models must be defined within a RAVEN \xmlNode{ExternalModel}. This
XML node accepts the following attributes:
\begin{itemize}
	\item \xmlAttr{name}, \xmlDesc{required string attribute}, user-defined identifier of this model.
	\nb As with other objects, this identifier can be used to reference this specific entity from other
	input blocks in the XML.
	\item \xmlAttr{subType}, \xmlDesc{required string attribute}, defines which of the sub-types should
	be used. For maintenance models, must use \xmlString{SR2ML.MaintenanceModel} as sub-type.
\end{itemize}
In the maintenance \xmlNode{ExternalModel} input block, the following XML sub-nodes are required:
\begin{itemize}
	\item \xmlNode{variable}, \xmlDesc{string, required parameter}. Comma-separated list of variable
	names. Each variable name needs to match a variable used/defined in the maintenance model or variable
	coming from other RAVEN entity (i.e. Samplers, DataObjects and Models).
	\nb For all the maintenance models, the following outputs variables would be available. If the user
	added these output variables in the node \xmlNode{variables}, these variables would be also available to
	for use anywhere in the RAVEN input to refer to the maintenance model output variables.
	\begin{itemize}
		\item \xmlString{avail}, variable that contains the calculated availability value
		\item \xmlString{unavail}, variable that contains the calculated unavailability value
	\end{itemize}
	\nb When the external model variables are defined, at run time, RAVEN initializes
	them and tracks their values during the simulation.
	\item \xmlNode{MaintenanceModel}, \xmlDesc{required parameter}. The node is used to define the maintenance
	model, and it contains the following required XML attribute:
	\begin{itemize}
		\item \xmlAttr{type}, \xmlDesc{required string attribute}, user-defined identifier of the maintenance model.
		\nb the types for different maintenance models can be found at the beginning of this section.
	\end{itemize}
\end{itemize}
In addition, if the user wants to use the \textbf{alias} system, the following XML block can be input:
\begin{itemize}
	\item \xmlNode{alias} \xmlDesc{string, optional field} specifies alias for
	any variable of interest in the input or output space for the ExternalModel.
	%
	These aliases can be used anywhere in the RAVEN input to refer to the ExternalModel
	variables.
	%
	In the body of this node the user specifies the name of the variable that the ExternalModel is
	going to use (during its execution).
	%
	The actual alias, usable throughout the RAVEN input, is instead defined in the
	\xmlAttr{variable} attribute of this tag.
	\\The user can specify aliases for both the input and the output space. As sanity check, RAVEN
	requires an additional required attribute \xmlAttr{type}. This attribute can be either ``input'' or ``output''.
	%
	\nb The user can specify as many aliases as needed.
	%
	\default{None}
\end{itemize}


\subsubsection{Operating Model}
For an operating model, the unavailability $u$ is calculated as: 
\begin{equation}
	u = lambda*Tr/(1.0+lambda*Tr) + Tpm/Tm
\end{equation}
where:
\begin{itemize}
  \item lambda: is the component failure rate
  \item Tr: mean time to repair
  \item Tpm: mean time to perform preventive maintenance
  \item Tm: preventive maintenance interval
\end{itemize}

Example XML:
\begin{lstlisting}[style=XML]
    <ExternalModel name="PMmodelOperating" subType="SR2ML.MaintenanceModel">
      <variables>lambda,Tm,avail,unavail</variables>
      <MaintenanceModel type="PMModel">
        <type>operating</type>
        <Tr>24</Tr>
        <Tpm>10</Tpm>
        <lambda>lambda</lambda>
        <Tm>Tm</Tm>
      </MaintenanceModel>
    </ExternalModel>
\end{lstlisting}

\subsubsection{Standby Model}
For an operating model, the unavailability $u$ is calculated as: 
\begin{equation}
  u = rho + 0.5*lambda*Ti + Tt/Ti + (rho+lambda*Ti)*Tr/Ti + Tpm/Tm
\end{equation}
where:
\begin{itemize}
  \item rho: failure probability per demand
  \item Ti: surveillance test interval
  \item Tr: mean time to repair
  \item Tt: test duration
  \item Tpm: mean time to perform preventive maintenance
  \item Tm: preventive maintenance interval
  \item lamb: component failure rate
\end{itemize}

Example XML:
\begin{lstlisting}[style=XML]
    <ExternalModel name="PMmodelStandby" subType="SR2ML.MaintenanceModel">
      <variables>lambda,Tm,Ti,avail,unavail</variables>
      <MaintenanceModel type="PMModel">
        <type>standby</type>
        <Tr>24</Tr>
        <Tpm>10</Tpm>
        <Tt>5</Tt>
        <rho>0.01</rho>
        <lambda>lambda</lambda>
        <Ti>Ti</Ti>
        <Tm>Tm</Tm>
      </MaintenanceModel>
    </ExternalModel>
\end{lstlisting}
