\section{Reliability Models}
\label{sec:ReliabilityModels}

\newcommand{\aliasRequiredParameterDescription}[1]
{
	\xmlNode{#1}, \xmlDesc{string or float, required parameter}. See above definition.
	If a string was provided, the reliability model would treat it as an input variable that came
	from other entity of RAVEN. In this case, the variable must be listed in the sub-node
	\xmlNode{variables} under \xmlNode{ExternalModel}.
}
\newcommand{\aliasOptionalParameterDescription}[1]
{
	\xmlNode{#1}, \xmlDesc{string or float, optional parameter}. See above definition.
	If a string was provided, the reliability model would treat it as an input variable that came
	from other entity of RAVEN. In this case, the variable must be listed in the sub-node
	\xmlNode{variables} under \xmlNode{ExternalModel}.
	\default{1.}
}

\textbf{Reliability Models/Functions} are the most frequently used in life data analysis
and reliability engineering. These models/functions give the probability of a component
operating for a certain amount of time without failure. As such, the reliability models
are function of time, in that every reliability value has an associated time value. In
other words, one must specify a time value with the desired reliability value. This degree
of flexibility makes the reliability model a much better reliability specification that the
mean time to failure (MTTF), which represents only one point along the entire reliability
model.

\subsection{The Probability Density and Cumulative Density Functions}
From probability and statistics, given a continuous random variable $X$, we denote:
\begin{itemize}
	\item The probability density function (pdf), as $f(x)$
	\item The cumulative density function (cdf), as $F(x)$.
\end{itemize}
If $x$ is a continuous random variable, then the probability of $x$ takes on a value in the
interval $[a,b]$ is the area under the pdf $f(x)$ from $a$ to $b$:
\begin{equation}
  P(a\leq x\leq b) = \int_{a}^{b} f(x)dx
\end{equation}
The cumulative distribution function is a function $F(x)$ of a random variable $x$, and is
defined for a number $x_0$ by:
\begin{equation}
  F(x_0) = P(x\leq x_0) = \int_{-\infty}^{x_0} f(s)ds
\end{equation}
That is, for a given value $x_0$, $F(x_0)$ is the probability that the observed value of $x$
would be, at most, $x_0$. The mathematical relationship between the pdf and cdf is given by:
\begin{equation}
  F(x) = \int_{-\infty}^{x} f(s)ds
\end{equation}
Conversely:
\begin{equation}
  f(x) = - \frac{dF(x)}{dx}
\end{equation}
The functions most commonly used in reliability engineering and life data analysis, namely the
reliability function and failure rate function, can be determined directly from the pdf definition,
or $f(t)$. Different distributions exist, such as Lognormal, Exponential, Weibull etc., and each of
them has a predefined $f(t)$. These distributions were formulated by statisticians, mathematicians
and engineers to mathematically model or represent certain behavior. Some distributions tend to better
represent life data and are most commonly referred to as lifetime distribution.

\subsection{The Reliability and Failure Rate Models}
Given the mathematical representation of a distribution, we can derive all functions needed
for reliability analysis, i.e. reliability models/functions. This would only depend on the value of $t$
after the value of the distribution parameters are estimated from data.
Now, let $T$ be the random variable defining the lifetime of the component with cdf $F(t)$, which is the
time the component would operate before failure. The cdf $F(t)$ of the random variable $T$ is given by:
\begin{equation}
  F(t) = \int_{-\infty}^{t} f(T)dT
\end{equation}
If $F(t)$ is a differentiable function, then the pdf $f(t)$ is given by:
\begin{equation}
  f(t) = - \frac{dF(t)}{dt}
\end{equation}
The reliability function or survival function $R(t)$ of the component is given by:
\begin{equation}
  R(t) = P(T>t) = 1 - P(T\leq t) = 1-F(t)
\end{equation}
This is the probability that the component would operate after time t, sometimes called survival probability.
The failure rate of a system during the interval $[t,t+\Delta t]$ is the probability that a failure per
unit time occurs in the interval, given that a failure has not occurred prior to t, the beginning of the
interval. The failure rate function (i.e. instantaneous failure rate, conditional failure rate) of hazard
function is defined as the limit of the failure rate as the interval approaches zero:
\begin{equation}
  \lambda (t)= \lim_{\Delta t\rightarrow 0} \frac{F(t+\Delta t) - F(t)}{\Delta tR(t)}
	 = \frac{1}{R(t)} \lim_{\Delta t\rightarrow 0} \frac{F(t+\Delta t) - F(t)}{\Delta t}
	 = \frac{1}{R(t)}\frac{dF(t)}{dt} = \frac{f(t)}{R(t)}
\end{equation}
The failure rate function is the rate of change of the conditional probability of failure at time $t$.
It measures the likelihood that a component that has operated up until time $t$ fails in the next
instant of time.
Generally $\lambda (t)$ is the one tabulated because it is measured experimentally and because it tends to
vary less rapidly with time than the other parameters. When $\lambda (t)$ is given, all other three
parameters $F(t)$, $f(t)$, $R(t)$ can be computed as follows:
\begin{equation}
  R(t) = \exp(-\int_{0}^{t} \lambda (s)ds)
\end{equation}
\begin{equation}
	f(t) = \lambda (t)R(t) = \lambda (t)\exp(-\int_{0}^{t} \lambda (s)ds)
\end{equation}
\begin{equation}
 	F(t) = 1 - R(t) = 1 - \exp(-\int_{0}^{t} \lambda (s)ds)
\end{equation}
The mean time between failure (MTBF) can be obtained by finding the expected value of the random variable
$T$, time to failure. Hence
\begin{equation}
  MTBF = E(T) = \int_{0}^{\infty} tf(t)dt = \int_{0}^{\infty} R(t)dt
\end{equation}

\subsection{The Lifetime Distributions or Aging Models}
We would consider several of the most useful reliability models based on different probability
distributions for describing failure of continuous operating devices, including:
\begin{itemize}
	\item Exponential, i.e. model \xmlAttr{type} is \xmlString{exponential}
	\item Erlangian, i.e. model \xmlAttr{type} is \xmlString{erlangian}
	\item Gamma, i.e. model \xmlAttr{type} is \xmlString{gamma}
	\item Lognormal, i.e. model \xmlAttr{type} is \xmlString{lognorm}
	\item Fatigue Life, i.e. model \xmlAttr{type} is \xmlString{fatiguelife}
	\item Weibull, i.e. model \xmlAttr{type} is \xmlString{weibull}
	\item Exponential Weibull, i.e. model \xmlAttr{type} is \xmlString{exponweibull}
	\item Bathtub, i.e. model \xmlAttr{type} is \xmlString{bathtub}
	\item Power Law, i.e. model \xmlAttr{type} is \xmlString{powerlaw}
	\item Log Linear, i.e. model \xmlAttr{type} is \xmlString{loglinear}.
\end{itemize}

The specifications of these models must be defined within a RAVEN \xmlNode{ExternalModel}. This
XML node accepts the following attributes:
\begin{itemize}
	\item \xmlAttr{name}, \xmlDesc{required string attribute}, user-defined identifier of this model.
	\nb As with other objects, this identifier can be used to reference this specific entity from other
	input blocks in the XML.
	\item \xmlAttr{subType}, \xmlDesc{required string attribute}, defines which of the sub-types should
	be used. For reliability models, must use \xmlString{SR2ML.ReliabilityModel} as sub-type.
\end{itemize}
In the reliability \xmlNode{ExternalModel} input block, the following XML sub-nodes are required:
\begin{itemize}
	\item \xmlNode{variable}, \xmlDesc{string, required parameter}. Comma-separated list of variable
	names. Each variable name needs to match a variable used/defined in the reliability model or variable
	coming from other RAVEN entity (i.e. Samplers, DataObjects and Models).
	\nb For all the reliability models, the following outputs variables would be available. If the user
	added these output variables in the node \xmlNode{variables}, these variables would be also available to
	for use anywhere in the RAVEN input to refer to the reliability model output variables.
	\begin{itemize}
		\item \xmlString{pdf\_f}, variable contains the calculated pdf value or values at given time instance(s),
		i.e. a series of times.
		\item \xmlString{cdf\_F}, variable contains the calculated cdf value or values at given time instance(s),
		i.e. a series of times.
		\item \xmlString{rdf\_R}, variable contains the calculated reliability function (rdf) value or values at
		given time instance(s), i.e. a series of times.
		\item \xmlString{frf\_h}, variable contains the calculated failure rate function (frf) value or values
		at given time instance(s), i.e. a series of times.
	\end{itemize}
	\nb When the external model variables are defined, at run time, RAVEN initializes
	them and tracks their values during the simulation.
	\item \xmlNode{ReliabilityModel}, \xmlDesc{required parameter}. The node is used to define the reliability
	model, and it contains the following required XML attribute:
	\begin{itemize}
		\item \xmlAttr{type}, \xmlDesc{required string attribute}, user-defined identifier of the reliability model.
		\nb the types for different reliability models can be found at the beginning of this section.
	\end{itemize}
	In addition, this node accepts several different sub-nodes representing the model parameters depending on the
	\xmlAttr{type} of the reliability model. The common sub-nodes for all reliability models are:
	\begin{itemize}
		\item \xmlNode{Tm}, \xmlDesc{string or float or comma-separated float, required parameter}. Time instance(s)
		that the reliability models would use to compute pdf, cdf, rdf, and frf values. If a string was provided,
		the reliability model would treat it as an input variable that came from entities of RAVEN. In this
		case, the variable must be listed in the sub-node \xmlNode{variables} under \xmlNode{ExternalModel}.
		\item \xmlNode{Td}, \xmlDesc{string or float, optional parameter}. Time that the reliability models start to be active.
		If a string was provided, the reliability model would treat it as an input variable that came
		from entities of RAVEN. In this case, the variable must be listed in the sub-node \xmlNode{variables}
		under \xmlNode{ExternalModel}.
		\default{0.}
	\end{itemize}
\end{itemize}
In addition, if the user wants to use the \textbf{alias} system, the following XML block can be input:
\begin{itemize}
	\item \xmlNode{alias} \xmlDesc{string, optional field} specifies alias for
	any variable of interest in the input or output space for the ExternalModel.
	%
	These aliases can be used anywhere in the RAVEN input to refer to the ExternalModel
	variables.
	%
	In the body of this node the user specifies the name of the variable that the ExternalModel is
	going to use (during its execution).
	%
	The actual alias, usable throughout the RAVEN input, is instead defined in the
	\xmlAttr{variable} attribute of this tag.
	\\The user can specify aliases for both the input and the output space. As sanity check, RAVEN
	requires an additional required attribute \xmlAttr{type}. This attribute can be either ``input'' or ``output''.
	%
	\nb The user can specify as many aliases as needed.
	%
	\default{None}
\end{itemize}

Example XML (Bathtub Reliability Model):
\begin{lstlisting}[style=XML]
	<ExternalModel name="bathtub" subType="SR2ML.ReliabilityModel">
		<variables>cdf_F, pdf_f, rdf_R, frf_h, tm</variables>
		<!-- xml portion for this plugin only -->
		<ReliabilityModel type="bathtub">
			<!-- scale parameter -->
			<beta>1.</beta>
			<theta>1.0</theta>
			<!-- mission time -->
			<Tm>tm</Tm>
			<!-- shape parameter -->
			<alpha>1.0</alpha>
			<rho>0.5</rho>
			<!-- weight parameter -->
			<c>0.5</c>
		</ReliabilityModel>
		<!--  alias can be used to represent any input/output variables -->
		<alias variable='bathtub_F' type='output'>cdf_F</alias>
		<alias variable='bathtub_f' type='output'>pdf_f</alias>
		<alias variable='bathtub_R' type='output'>rdf_R</alias>
		<alias variable='bathtub_h' type='output'>frf_h</alias>
	</ExternalModel>
\end{lstlisting}

\subsubsection{The Lognormal Model}
The probability density function of the lognormal is given by
\begin{equation}
	f(T_m) = \frac{1}{\alpha T_m\sqrt{2\pi}}\exp\left(-\frac{1}{2}\left(\frac{\ln{\frac{T_m-T_d}{\beta}}}{\alpha}\right)^2\right)
\end{equation}
where $T_m\geq T_d$, $T_d, \alpha, \beta >0$, and $\beta$ is the scale parameter, $\alpha$ is the shape
parameter, and $T_d$ is the location parameter.
This model accepts the following additional sub-nodes:
\begin{itemize}
	\item \aliasRequiredParameterDescription{alpha}
	\item \aliasOptionalParameterDescription{beta}
\end{itemize}

Example XML:
\begin{lstlisting}[style=XML]
	<ExternalModel name="lognorm" subType="SR2ML.ReliabilityModel">
		<variables>cdf_F, pdf_f, rdf_R, frf_h, tm</variables>
		<ReliabilityModel type="lognorm">
			<!-- scale parameter -->
			<beta>1.</beta>
			<!-- mission time -->
			<Tm>tm</Tm>
			<!-- shape parameter -->
			<alpha>1.</alpha>
		</ReliabilityModel>
	</ExternalModel>
\end{lstlisting}

\subsubsection{The Exponential Model}
The probability density function of the Exponential is given by
\begin{equation}
	f(T_m) = \lambda\exp\left(-\lambda\left(T_m-T_d\right)\right)
\end{equation}
where $T_m\geq T_d$, $T_d, \lambda >0$, $\lambda$ is the mean failure rate or the inverse of scale parameter,
and $T_d$ is the location parameter.
This model accepts the following additional sub-nodes:
\begin{itemize}
	\item \aliasRequiredParameterDescription{lambda}
\end{itemize}

Example XML:
\begin{lstlisting}[style=XML]
	<ExternalModel name="exponential" subType="SR2ML.ReliabilityModel">
		<variables>cdf_F, pdf_f, rdf_R, frf_h, tm</variables>
		<!-- xml portion for this plugin only -->
		<ReliabilityModel type="exponential">
			<!-- mean failure rate -->
			<lambda>1.</lambda>
			<!-- mission time -->
			<Tm>tm</Tm>
		</ReliabilityModel>
	</ExternalModel>
\end{lstlisting}

\subsubsection{The Weibull Model}
The probability density function of the Weibull is given by
\begin{equation}
	f(T_m) = \frac{\alpha}{\beta}\left(\frac{T_m-T_d}{\beta}\right)^{\alpha-1}\exp\left(-\left(\frac{T_m-T_d}{\beta}\right)^\alpha\right)
\end{equation}
where $T_m\geq T_d$, $T_d, \alpha, \beta >0$, and $\beta$ is the scale parameter, $\alpha$ is the shape
parameter, and $T_d$ is the location parameter.
This model accepts the following additional sub-nodes:
\begin{itemize}
	\item \aliasRequiredParameterDescription{alpha}
	\item \aliasOptionalParameterDescription{beta}
\end{itemize}

Example XML:
\begin{lstlisting}[style=XML]
	<ExternalModel name="weibull" subType="SR2ML.ReliabilityModel">
		<variables>cdf_F, pdf_f, rdf_R, frf_h, tm</variables>
		<!-- xml portion for this plugin only -->
		<ReliabilityModel type="weibull">
			<!-- scale parameter -->
			<beta>1.</beta>
			<!-- mission time -->
			<Tm>tm</Tm>
			<!-- time delay -->
			<Td>2.0</Td>
			<!-- shape parameter -->
			<alpha>1.0</alpha>
		</ReliabilityModel>
	</ExternalModel>
\end{lstlisting}

\subsubsection{The Erlangian Model}
The probability density function of the Erlangian is given by
\begin{equation}
	f(T_m) = \frac{\lambda\left(\lambda T_m\right)^{k-1}\exp\left(-\lambda T_m\right)}{\left(k-1\right)!}
\end{equation}
where $T_m\geq T_d$, $T_d, \lambda >0$, and $\lambda$ is the inverse of scale parameter, $k$ is positive integer
that control the shape, and $T_d$ is the location parameter.
This model accepts the following additional sub-nodes:
\begin{itemize}
	\item \aliasRequiredParameterDescription{alpha}
	\item \aliasOptionalParameterDescription{k}
	\nb $k$ is a positive integer. If a float was provided, a warning would be raised.
\end{itemize}

Example XML:
\begin{lstlisting}[style=XML]
	<ExternalModel name="erlangian" subType="SR2ML.ReliabilityModel">
		<variables>cdf_F, pdf_f, rdf_R, frf_h, tm</variables>
		<!-- xml portion for this plugin only -->
		<ReliabilityModel type="erlangian">
			<!-- mean failure rate -->
			<lambda>0.1</lambda>
			<!-- mission time -->
			<Tm>tm</Tm>
			<!-- shape parameter -->
			<k>2</k>
		</ReliabilityModel>
	</ExternalModel>
\end{lstlisting}


\subsubsection{The Gamma Model}
The probability density function of the Gamma is given by
\begin{equation}
	f(T_m) = \frac{\beta \left(\beta \left(T_m-T_d\right)\right)^{\alpha-1}\exp\left(-\beta\left(T_m-T_d\right)\right)}{\Gamma \left(\alpha\right)}
\end{equation}
where $T_m\geq T_d$, $T_d, \alpha, \beta >0$, and $\beta$ is the inverse of scale parameter, $\alpha$ is the shape
parameter, and $T_d$ is the location parameter.
This model accepts the following additional sub-nodes:
\begin{itemize}
	\item \aliasRequiredParameterDescription{alpha}
	\item \aliasOptionalParameterDescription{beta}
\end{itemize}

Example XML:
\begin{lstlisting}[style=XML]
	<ExternalModel name="gamma" subType="SR2ML.ReliabilityModel">
		<variables>cdf_F, pdf_f, rdf_R, frf_h, tm</variables>
		<!-- xml portion for this plugin only -->
		<ReliabilityModel type="gamma">
			<!-- rate parameter -->
			<beta>0.1</beta>
			<!-- mission time -->
			<Tm>tm</Tm>
			<!-- shape parameter -->
			<alpha>2.</alpha>
		</ReliabilityModel>
	</ExternalModel>
\end{lstlisting}


\subsubsection{The Fatigue Life Model (Birnbaum-Saunders)}
The probability density function of the fatigue life is given by
\begin{equation}
	f(T_m) = \frac{\frac{T_m-T_d}{\beta}+1}{2\alpha\sqrt{2\pi\left(\frac{T_m-T_d}{\beta}\right)^3}}
				 \exp\left(-\frac{\left(\frac{T_m-T_d}{\beta}-1\right)^2}{2\left(\frac{T_m-T_d}{\beta}\right)\alpha^2}\right)
\end{equation}
where $T_m\geq T_d$, $T_d, \alpha, \beta >0$, and $\beta$ is the scale parameter, $\alpha$ is the shape
parameter, and $T_d$ is the location parameter.
This model accepts the following additional sub-nodes:
\begin{itemize}
	\item \aliasRequiredParameterDescription{alpha}
	\item \aliasOptionalParameterDescription{beta}
\end{itemize}

Example XML:
\begin{lstlisting}[style=XML]
	<ExternalModel name="fatiguelife" subType="SR2ML.ReliabilityModel">
		<variables>cdf_F, pdf_f, rdf_R, frf_h, tm</variables>
		<!-- xml portion for this plugin only -->
		<ReliabilityModel type="fatiguelife">
			<!-- scale parameter -->
			<beta>1.</beta>
			<!-- mission time -->
			<Tm>tm</Tm>
			<!-- shape parameter -->
			<alpha>1.0</alpha>
		</ReliabilityModel>
	</ExternalModel>
\end{lstlisting}


\subsubsection{The Exponentiated Weibull Model}
The probability density function of the exponentiated Weibull is given by
\begin{equation}
	f(T_m) = \gamma\alpha\left(1-\exp\left(-\left(\frac{T_m-T_d}{\beta}\right)^\alpha\right)\right)^{\gamma-1}
				 \left(\frac{T_m-T_d}{\beta}\right)^{\alpha-1}\exp\left(-\left(\frac{T_m-T_d}{\beta}\right)^\alpha\right)
\end{equation}
where $T_m\geq T_d$, $T_d, \alpha, \beta, \gamma>0$, and $\beta$ is the scale parameter, $\alpha$ and $\gamma$ is the shape
parameter, and $T_d$ is the location parameter.
This model accepts the following additional sub-nodes:
\begin{itemize}
	\item \aliasRequiredParameterDescription{alpha}
	\item \aliasOptionalParameterDescription{beta}
	\item \aliasRequiredParameterDescription{gamma}
\end{itemize}

Example XML:
\begin{lstlisting}[style=XML]
	<ExternalModel name="exponweibull" subType="SR2ML.ReliabilityModel">
		<variables>cdf_F, pdf_f, rdf_R, frf_h, tm</variables>
		<!-- xml portion for this plugin only -->
		<ReliabilityModel type="exponweibull">
			<!-- scale parameter -->
			<beta>1.</beta>
			<!-- mission time -->
			<Tm>tm</Tm>
			<!-- time delay -->
			<Td>2.0</Td>
			<!-- shape parameter -->
			<alpha>1.0</alpha>
			<gamma>0.5</gamma>
		</ReliabilityModel>
	</ExternalModel>
\end{lstlisting}


\subsubsection{The Bathtub Model}
The reliability function is given by:
\begin{equation}
  R(T_m) = \exp\left(-c\beta\left(\frac{T_m-T_d}{\beta}\right)^\alpha -(1-c)\left(\exp\left(\frac{T_m-T_d}{\theta}\right)^\rho -1\right)\right)
\end{equation}
The failure rate function is given by:
\begin{equation}
  \lambda(T_m) = c\alpha\left(\frac{T_m-T_d}{\beta}\right)^{\alpha-1}+(1-c)\rho\left(\frac{T_m-T_d}{\theta}\right)^{\rho-1}
\end{equation}
The probability density function of the Bathtub is given by
\begin{equation}
	f(T_m) = \lambda(T_m) R(T_m)
\end{equation}
where $T_m\geq T_d$, $T_d, \alpha, \beta, \theta, \rho, c >0$, and $\beta, \theta$ are the scale parameters,
$\alpha, \rho$ are the shape parameters, $c \in [0,1]$ is the weight parameter, and $T_d$ is the location parameter.
This model accepts the following additional sub-nodes:
\begin{itemize}
	\item \aliasRequiredParameterDescription{alpha}
	\item \aliasOptionalParameterDescription{beta}
	\item \aliasOptionalParameterDescription{theta}
	\item \aliasOptionalParameterDescription{rho}
	\item \aliasOptionalParameterDescription{c}
\end{itemize}

Example XML:
\begin{lstlisting}[style=XML]
	<ExternalModel name="bathtub" subType="SR2ML.ReliabilityModel">
		<variables>cdf_F, pdf_f, rdf_R, frf_h, tm</variables>
		<!-- xml portion for this plugin only -->
		<ReliabilityModel type="bathtub">
			<!-- scale parameter -->
			<beta>1.</beta>
			<theta>1.0</theta>
			<!-- mission time -->
			<Tm>tm</Tm>
			<!-- shape parameter -->
			<alpha>1.0</alpha>
			<rho>0.5</rho>
			<!-- weight parameter -->
			<c>0.5</c>
		</ReliabilityModel>
	</ExternalModel>
\end{lstlisting}


\subsubsection{The Power Law Model for Failure Rate Function}
The hazard rate satisfies a power law as a function of time:
\begin{equation}
	\lambda(T_m) = \lambda + \alpha(T_m-T_d)^\beta
\end{equation}
where $T_m\geq T_d$, $T_d, \alpha, \beta, \lambda >0$,
and $T_d$ is the location parameter.
This model accepts the following additional sub-nodes:
\begin{itemize}
	\item \aliasOptionalParameterDescription{alpha}
	\item \aliasOptionalParameterDescription{beta}
	\item \aliasOptionalParameterDescription{lambda}
\end{itemize}

Example XML:
\begin{lstlisting}[style=XML]
	<ExternalModel name="powerlaw" subType="SR2ML.ReliabilityModel">
		<variables>cdf_F, pdf_f, rdf_R, frf_h, tm</variables>
		<!-- xml portion for this plugin only -->
		<ReliabilityModel type="powerlaw">
			<beta>1.0</beta>
			<alpha>1.0</alpha>
			<lambda>0.5</lambda>
			<Tm>tm</Tm>
		</ReliabilityModel>
	</ExternalModel>
\end{lstlisting}


\subsubsection{The Log Linear Model for Failure Rate Function}
The hazard rate satisfies a exponential law as a function of time:
\begin{equation}
	\lambda(T_m) = \exp\left(\alpha+\beta(T_m-T_d)\right)
\end{equation}
where $T_m\geq T_d$, $T_d, \alpha, \beta >0$, and $T_d$ is the location parameter.
This model accepts the following additional sub-nodes:
\begin{itemize}
	\item \aliasOptionalParameterDescription{alpha}
	\item \aliasOptionalParameterDescription{beta}
\end{itemize}

Example XML:
\begin{lstlisting}[style=XML]
	<ExternalModel name="loglinear" subType="SR2ML.ReliabilityModel">
		<variables>cdf_F, pdf_f, rdf_R, frf_h, tm</variables>
		<!-- xml portion for this plugin only -->
		<ReliabilityModel type="loglinear">
			<beta>1.</beta>
			<alpha>1.</alpha>
			<Tm>tm</Tm>
		</ReliabilityModel>
	</ExternalModel>
\end{lstlisting}


\subsection{Reliability Models Reference Tests}
\begin{itemize}
	\item SR2ML/tests/reliabilityModel/test\_bathtub.xml
  \item SR2ML/tests/reliabilityModel/test\_erlangian.xml
	\item SR2ML/tests/reliabilityModel/test\_expon.xml
  \item SR2ML/tests/reliabilityModel/test\_exponweibull.xml
	\item SR2ML/tests/reliabilityModel/test\_fatiguelife.xml
  \item SR2ML/tests/reliabilityModel/test\_gamma.xml
	\item SR2ML/tests/reliabilityModel/test\_loglinear.xml
  \item SR2ML/tests/reliabilityModeltest\_lognorm.xml
	\item SR2ML/tests/reliabilityModel/test\_normal.xml
  \item SR2ML/tests/reliabilityModel/test\_powerlaw.xml
	\item SR2ML/tests/reliabilityModeltest\_weibull.xml
	\item SR2ML/tests/reliabilityModel/test\_time\_dep\_ensemble\_reliability.xml.
\end{itemize}
