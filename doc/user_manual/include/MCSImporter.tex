\section{MCSImporter}
\label{MCSimporterPP}

The \textbf{MCSImporter} post-processor has been designed to import Minimal Cut Sets (MCSs) into RAVEN.
This post-processor reads a csv file which contain the list of MCSs and it save this list as a DataObject
(i.e.,  a PointSet).
The csv file is composed by three columns; the first contains the ID number of the MCS, the second one contains
the MCS probability value, the third one lists all the Basic Events contained in the MCS.
An example of csv file is shown in Table~\ref{MCScsv}.

\begin{table}[h]
    \centering
    \caption{Example of csv file which contains four MCSs.}
    \label{MCScsv}
	\begin{tabular}{c  c  c}
		\hline
		ID, & Prob, & MCS, \\
		\hline
		1.,  &  1.8E-2, &  D  \\
		2.,  &  4.0E-3, &  B \\
		3.,  &  3.0E-4, &  A,C  \\
		4.,  &  2.1E-5, &  E,C \\
		\hline
	\end{tabular}
\end{table}

The PointSet is structured to include all Basic Event, the MCS ID, the MCS probability, and the outcome of such MCS
(always set to 1).
MCS ID and MCS probability are copied directly from the csv file.
For each MCS, the Basic Events can have two possible values:
  \begin{itemize}
    \item  0: Basic Event is not included in the MCS
    \item  1: Basic Event is included in the MCS
  \end{itemize}
The PointSet generated from the csv file of Table~\ref{MCScsv} is shown in Table~\ref{PointSetMCSExpandFalse}.
\begin{table}[h]
    \centering
    \caption{PointSet generated by RAVEN for the list of MCSs shown in Table~\ref{MCScsv}.}
    \label{PointSetMCSExpandFalse}
	\begin{tabular}{c | c | c | c | c | c | c | c }
		\hline
		A & B & C & D & E & MCS\_ID & probability & out \\
		\hline
		0 & 0 & 0 & 1 & 0 & 1 & 1.8E-2 & 1 \\
		0 & 1 & 0 & 0 & 0 & 2 & 4.0E-3 & 1 \\
		1 & 0 & 1 & 0 & 0 & 3 & 3.0E-4 & 1 \\
		0 & 0 & 1 & 0 & 1 & 4 & 4.0E-3 & 1 \\
		\hline
	\end{tabular}
\end{table}

%
\ppType{MCSImporter}{MCSImporter}
%
\begin{itemize}
  \item  \xmlNode{expand},\xmlDesc{bool, required parameter}, expand the set of Basic Events by including all PRA Basic Events
  and not only the once listed in the MCSs
  \item  \xmlNode{BElistColumn},\xmlDesc{string, optional parameter}, if expand is set to True, then this node contains the
  column of the csv file which contains all the PRA Basic Events
  \item \xmlNode{fileFrom}, \xmlDesc{string, optional parameter}, either `None' or `saphire', indicates where the MCSs file is coming from. Currently,
  we only support normal csv file and file generated by Saphire code. Default is `None', which means the user need to provide
  the normal csv file as mentioned above.
\end{itemize}

\textbf{Example:}
\begin{lstlisting}[style=XML,morekeywords={anAttribute},caption=MCS Importer input example (no expand)., label=lst:MCS_PP_InputExample]
  <Files>
    <Input name="MCSlistFile" type="MCSlist">MCSlist.csv</Input>
  </Files>

  <Models>
    <PostProcessor name="MCSImporter" subType="SR2ML.MCSImporter">
      <expand>False</expand>
    </PostProcessor>
  </Models>

  <Steps>
    <PostProcess name="import">
      <Input   class="Files"        type="MCSlist"         >MCSlistFile</Input>
      <Model   class="Models"       type="PostProcessor"   >MCSImporter</Model>
      <Output  class="DataObjects"  type="PointSet"        >MCS_PS</Output>
    </PostProcess>
  </Steps>

  <DataObjects>
    <PointSet name="MCS_PS">
      <Input>A,B,C,D,E</Input>
      <Output>MCS_ID,probability,out</Output>
    </PointSet>
  </DataObjects>
\end{lstlisting}

\textbf{Example:}
\begin{lstlisting}[style=XML,morekeywords={anAttribute},caption=MCS Importer input example (expanded)., label=lst:MCS_PP_InputExample]
  <Files>
    <Input name="MCSlistFile" type="MCSlist">MCSlist.csv</Input>
    <Input name="BElistFile"  type="BElist" >BElist.csv</Input>
  </Files>

  <Models>
    <PostProcessor name="MCSImporter" subType="SR2ML.MCSImporter">
      <expand>False</expand>
    </PostProcessor>
  </Models>

  <Steps>
    <PostProcess name="import">
      <Input   class="Files"        type="MCSlist"         >MCSlistFile</Input>
      <Input   class="Files"        type="BElist"          >BElistFile</Input>
      <Model   class="Models"       type="PostProcessor"   >MCSimporter</Model>
      <Output  class="DataObjects"  type="PointSet"        >MCS_PS</Output>
    </PostProcess>
  </Steps>

  <DataObjects>
    <PointSet name="MCS_PS">
      <Input>A,B,C,D,E,F,G</Input>
      <Output>MCS_ID,probability,out</Output>
    </PointSet>
  </DataObjects>
\end{lstlisting}
