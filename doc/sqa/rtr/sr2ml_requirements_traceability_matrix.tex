%
% This is an example LaTeX file which uses the SANDreport class file.
% It shows how a SAND report should be formatted, what sections and
% elements it should contain, and how to use the SANDreport class.
% It uses the LaTeX article class, but not the strict option.
% ItINLreport uses .eps logos and files to show how pdflatex can be used
%
% Get the latest version of the class file and more at
%    http://www.cs.sandia.gov/~rolf/SANDreport
%
% This file and the SANDreport.cls file are based on information
% contained in "Guide to Preparing {SAND} Reports", Sand98-0730, edited
% by Tamara K. Locke, and the newer "Guide to Preparing SAND Reports and
% Other Communication Products", SAND2002-2068P.
% Please send corrections and suggestions for improvements to
% Rolf Riesen, Org. 9223, MS 1110, rolf@cs.sandia.gov
%
\documentclass[pdf,12pt]{../sdd/INLreport}
% pslatex is really old (1994).  It attempts to merge the times and mathptm packages.
% My opinion is that it produces a really bad looking math font.  So why are we using it?
% If you just want to change the text font, you should just \usepackage{times}.
% \usepackage{pslatex}
\usepackage{times}
%\usepackage{longtable}
\usepackage[FIGBOTCAP,normal,bf,tight]{subfigure}
\usepackage{amsmath}
\usepackage{tabularx}
\usepackage{ltablex}
\usepackage{amssymb}
\usepackage[labelfont=bf]{caption}
\usepackage{pifont}
\usepackage{enumerate}
\usepackage{listings}
\usepackage{fullpage}
\usepackage{xcolor}          % Using xcolor for more robust color specification
\usepackage{ifthen}          % For simple checking in newcommand blocks
\usepackage{textcomp}
%\usepackage{authblk}         % For making the author list look prettier
%\renewcommand\Authsep{,~\,}

% Custom colors
\definecolor{deepblue}{rgb}{0,0,0.5}
\definecolor{deepred}{rgb}{0.6,0,0}
\definecolor{deepgreen}{rgb}{0,0.5,0}
\definecolor{forestgreen}{RGB}{34,139,34}
\definecolor{orangered}{RGB}{239,134,64}
\definecolor{darkblue}{rgb}{0.0,0.0,0.6}
\definecolor{gray}{rgb}{0.4,0.4,0.4}

\lstset {
  basicstyle=\ttfamily,
  frame=single
}

\setcounter{secnumdepth}{5}
\lstdefinestyle{XML} {
    language=XML,
    extendedchars=true,
    breaklines=true,
    breakatwhitespace=true,
%    emph={name,dim,interactive,overwrite},
    emphstyle=\color{red},
    basicstyle=\ttfamily,
%    columns=fullflexible,
    commentstyle=\color{gray}\upshape,
    morestring=[b]",
    morecomment=[s]{<?}{?>},
    morecomment=[s][\color{forestgreen}]{<!--}{-->},
    keywordstyle=\color{cyan},
    stringstyle=\ttfamily\color{black},
    tagstyle=\color{darkblue}\bf\ttfamily,
    morekeywords={name,type},
%    morekeywords={name,attribute,source,variables,version,type,release,x,z,y,xlabel,ylabel,how,text,param1,param2,color,label},
}
\lstset{language=python,upquote=true}

\usepackage{titlesec}
\newcommand{\sectionbreak}{\clearpage}
\setcounter{secnumdepth}{4}

%\titleformat{\paragraph}
%{\normalfont\normalsize\bfseries}{\theparagraph}{1em}{}
%\titlespacing*{\paragraph}
%{0pt}{3.25ex plus 1ex minus .2ex}{1.5ex plus .2ex}

%%%%%%%% Begin comands definition to input python code into document
\usepackage[utf8]{inputenc}

% Default fixed font does not support bold face
\DeclareFixedFont{\ttb}{T1}{txtt}{bx}{n}{9} % for bold
\DeclareFixedFont{\ttm}{T1}{txtt}{m}{n}{9}  % for normal

\usepackage{listings}

% Python style for highlighting
\newcommand\pythonstyle{\lstset{
language=Python,
basicstyle=\ttm,
otherkeywords={self, none, return},             % Add keywords here
keywordstyle=\ttb\color{deepblue},
emph={MyClass,__init__},          % Custom highlighting
emphstyle=\ttb\color{deepred},    % Custom highlighting style
stringstyle=\color{deepgreen},
frame=tb,                         % Any extra options here
showstringspaces=false            %
}}


% Python environment
\lstnewenvironment{python}[1][]
{
\pythonstyle
\lstset{#1}
}
{}

% Python for external files
\newcommand\pythonexternal[2][]{{
\pythonstyle
\lstinputlisting[#1]{#2}}}

\lstnewenvironment{xml}
{}
{}

% Python for inline
\newcommand\pythoninline[1]{{\pythonstyle\lstinline!#1!}}

% Named Colors for the comments below (Attempted to match git symbol colors)
\definecolor{RScolor}{HTML}{8EB361}  % Sonat (adjusted for clarity)
\definecolor{DPMcolor}{HTML}{E28B8D} % Dan
\definecolor{JCcolor}{HTML}{82A8D9}  % Josh (adjusted for clarity)
\definecolor{AAcolor}{HTML}{8D7F44}  % Andrea
\definecolor{CRcolor}{HTML}{AC39CE}  % Cristian
\definecolor{RKcolor}{HTML}{3ECC8D}  % Bob (adjusted for clarity)
\definecolor{DMcolor}{HTML}{276605}  % Diego (adjusted for clarity)
\definecolor{PTcolor}{HTML}{990000}  % Paul

\def\DRAFT{} % Uncomment this if you want to see the notes people have been adding
% Comment command for developers (Should only be used under active development)
\ifdefined\DRAFT
  \newcommand{\nameLabeler}[3]{\textcolor{#2}{[[#1: #3]]}}
\else
  \newcommand{\nameLabeler}[3]{}
\fi
\newcommand{\alfoa}[1] {\nameLabeler{Andrea}{AAcolor}{#1}}
\newcommand{\cristr}[1] {\nameLabeler{Cristian}{CRcolor}{#1}}
\newcommand{\mandd}[1] {\nameLabeler{Diego}{DMcolor}{#1}}
\newcommand{\maljdan}[1] {\nameLabeler{Dan}{DPMcolor}{#1}}
\newcommand{\cogljj}[1] {\nameLabeler{Josh}{JCcolor}{#1}}
\newcommand{\bobk}[1] {\nameLabeler{Bob}{RKcolor}{#1}}
\newcommand{\senrs}[1] {\nameLabeler{Sonat}{RScolor}{#1}}
\newcommand{\talbpaul}[1] {\nameLabeler{Paul}{PTcolor}{#1}}
% Commands for making the LaTeX a bit more uniform and cleaner
\newcommand{\TODO}[1]    {\textcolor{red}{\textit{(#1)}}}
\newcommand{\xmlAttrRequired}[1] {\textcolor{red}{\textbf{\texttt{#1}}}}
\newcommand{\xmlAttr}[1] {\textcolor{cyan}{\textbf{\texttt{#1}}}}
\newcommand{\xmlNodeRequired}[1] {\textcolor{deepblue}{\textbf{\texttt{<#1>}}}}
\newcommand{\xmlNode}[1] {\textcolor{darkblue}{\textbf{\texttt{<#1>}}}}
\newcommand{\xmlString}[1] {\textcolor{black}{\textbf{\texttt{'#1'}}}}
\newcommand{\xmlDesc}[1] {\textbf{\textit{#1}}} % Maybe a misnomer, but I am
                                                % using this to detail the data
                                                % type and necessity of an XML
                                                % node or attribute,
                                                % xmlDesc = XML description
\newcommand{\default}[1]{~\\*\textit{Default: #1}}
\newcommand{\nb} {\textcolor{deepgreen}{\textbf{~Note:}}~}

%%%%%%%% End comands definition to input python code into document

%\usepackage[dvips,light,first,bottomafter]{draftcopy}
%\draftcopyName{Sample, contains no OUO}{70}
%\draftcopyName{Draft}{300}

% The bm package provides \bm for bold math fonts.  Apparently
% \boldsymbol, which I used to always use, is now considered
% obsolete.  Also, \boldsymbol doesn't even seem to work with
% the fonts used in this particular document...
\usepackage{bm}

% Define tensors to be in bold math font.
\newcommand{\tensor}[1]{{\bm{#1}}}

% Override the formatting used by \vec.  Instead of a little arrow
% over the letter, this creates a bold character.
\renewcommand{\vec}{\bm}

% Define unit vector notation.  If you don't override the
% behavior of \vec, you probably want to use the second one.
\newcommand{\unit}[1]{\hat{\bm{#1}}}
% \newcommand{\unit}[1]{\hat{#1}}

% Use this to refer to a single component of a unit vector.
\newcommand{\scalarunit}[1]{\hat{#1}}

% \toprule, \midrule, \bottomrule for tables
\usepackage{booktabs}

% \llbracket, \rrbracket
\usepackage{stmaryrd}

\usepackage{hyperref}
\hypersetup{
    colorlinks,
    citecolor=black,
    filecolor=black,
    linkcolor=black,
    urlcolor=black
}
%\usepackage[table,xcdraw]{xcolor}
\newcommand{\wiki}{\href{https://github.com/idaholab/raven/wiki}{RAVEN wiki}}

% Compress lists of citations like [33,34,35,36,37] to [33-37]
\usepackage{cite}

% If you want to relax some of the SAND98-0730 requirements, use the "relax"
% option. It adds spaces and boldface in the table of contents, and does not
% force the page layout sizes.
% e.g. \documentclass[relax,12pt]{SANDreport}
%
% You can also use the "strict" option, which applies even more of the
% SAND98-0730 guidelines. It gets rid of section numbers which are often
% useful; e.g. \documentclass[strict]{SANDreport}

% The INLreport class uses \flushbottom formatting by default (since
% it's intended to be two-sided document).  \flushbottom causes
% additional space to be inserted both before and after paragraphs so
% that no matter how much text is actually available, it fills up the
% page from top to bottom.  My feeling is that \raggedbottom looks much
% better, primarily because most people will view the report
% electronically and not in a two-sided printed format where some argue
% \raggedbottom looks worse.  If we really want to have the original
% behavior, we can comment out this line...
\raggedbottom
\setcounter{secnumdepth}{5} % show 5 levels of subsection
\setcounter{tocdepth}{5} % include 5 levels of subsection in table of contents

% ---------------------------------------------------------------------------- %
%
% Set the title, author, and date
%
\title{SR2ML Requirements Traceability Matrix}
%\author{%
%\begin{tabular}{c} Author 1 \\ University1 \\ Mail1 \\ \\
%Author 3 \\ University3 \\ Mail3 \end{tabular} \and
%\begin{tabular}{c} Author 2 \\ University2 \\ Mail2 \\ \\
%Author 4 \\ University4 \\ Mail4\\
%\end{tabular} }


\author{Congjian Wang, Diego Mandelli, Andrea Alfonsi}
 

% There is a "Printed" date on the title page of a SAND report, so
% the generic \date should [WorkingDir:]generally be empty.
\date{}


% ---------------------------------------------------------------------------- %
% Set some things we need for SAND reports. These are mandatory
%
\SANDnum{RAVEN-RTM}
\SANDprintDate{\today}
\SANDauthor{Congjian Wang, Diego Mandelli, Andrea Alfonsi}
\SANDreleaseType{Revision 0}

% ---------------------------------------------------------------------------- %
% Include the markings required for your SAND report. The default is "Unlimited
% Release". You may have to edit the file included here, or create your own
% (see the examples provided).
%
% \include{MarkOUO} % Not needed for unlimted release reports

\def\component#1{\texttt{#1}}

% ---------------------------------------------------------------------------- %
\newcommand{\systemtau}{\tensor{\tau}_{\!\text{SUPG}}}

% Added by Sonat
\usepackage{placeins}
\usepackage{array}

\newcolumntype{L}[1]{>{\raggedright\let\newline\\\arraybackslash\hspace{0pt}}m{#1}}
\newcolumntype{C}[1]{>{\centering\let\newline\\\arraybackslash\hspace{0pt}}m{#1}}
\newcolumntype{R}[1]{>{\raggedleft\let\newline\\\arraybackslash\hspace{0pt}}m{#1}}

% end added by Sonat
% ---------------------------------------------------------------------------- %
%
% Start the document
%

\begin{document}
    \maketitle

    % ------------------------------------------------------------------------ %
    % An Abstract is required for SAND reports
    %
%    \begin{abstract}
%    \input abstract
%    \end{abstract}


    % ------------------------------------------------------------------------ %
    % An Acknowledgement section is optional but important, if someone made
    % contributions or helped beyond the normal part of a work assignment.
    % Use \section* since we don't want it in the table of context
    %
%    \clearpage
%    \section*{Acknowledgment}



%	The format of this report is based on information found
%	in~\cite{Sand98-0730}.


    % ------------------------------------------------------------------------ %
    % The table of contents and list of figures and tables
    % Comment out \listoffigures and \listoftables if there are no
    % figures or tables. Make sure this starts on an odd numbered page
    %
    \cleardoublepage		% TOC needs to start on an odd page
    \tableofcontents
    %\listoffigures
    %\listoftables


    % ---------------------------------------------------------------------- %
    % An optional preface or Foreword
%    \clearpage
%    \section*{Preface}
%    \addcontentsline{toc}{section}{Preface}
%	Although muggles usually have only limited experience with
%	magic, and many even dispute its existence, it is worthwhile
%	to be open minded and explore the possibilities.


    % ---------------------------------------------------------------------- %
    % An optional executive summary
    %\clearpage
    %\section*{Summary}
    %\addcontentsline{toc}{section}{Summary}
    %\input{Summary.tex}
%	Once a certain level of mistrust and skepticism has
%	been overcome, magic finds many uses in todays science



%	and engineering. In this report we explain some of the
%	fundamental spells and instruments of magic and wizardry. We
%	then conclude with a few examples on how they can be used
%	in daily activities at national Laboratories.


    % ---------------------------------------------------------------------- %
    % An optional glossary. We don't want it to be numbered
%    \clearpage
%    \section*{Nomenclature}
%    \addcontentsline{toc}{section}{Nomenclature}
%    \begin{description}
%          \item[alohomoral]
%           spell to open locked doors and containers
%          \item[leviosa]
%           spell to levitate objects
%    \item[remembrall]
%           device to alert you that you have forgotten something
%    \item[wand]
%           device to execute spells
%    \end{description}


    % ---------------------------------------------------------------------- %
    % This is where the body of the report begins; usually with an Introduction
    %
    \SANDmain		% Start the main part of the report

\section{Introduction}
The \textbf{\textit{SR2ML}} plug-in is a generalized module for safety risk and reliability analysis within RAVEN.
\\The plug-in is aimed to perform structure, system and component safety risk and reliability analysis.
\\This document is aimed to report the traceability matrix between software requirements
(see SR2ML SRS) and requirement tests (tests that testify the module/plug-in is compliant
with respect its own requirements).

\subsection{Dependencies and Limitations}
The plug-in should be designed with the fewest possible constraints.
Ideally the  plug-in (in conjunction with RAVEN)
 should run on a wide variety of evolving hardware,
so it should follow well-adopted standards and guidelines. The software
 should run on any POSIX compliant system (including Windows POSIX
 emulators such as MinGW)l.
\\In order to be functional, \textit{\textbf{SR2ML}} depends on the following software/libraries.
\begin{itemize}
  \item RAVEN (\url{raven.inl.gov}) and all its dependencies (listed in ``RAVEN Software Design Description'' - SDD-513)
\end{itemize}


\section{References}

\begin{itemize}

  \item ASME NQA 1 2008 with the NQA-1a-2009 addenda, ``Quality Assurance Requirements for Nuclear Facility Applications,'' First Edition, August 31, 2009.
  \item ISO/IEC/IEEE 24765:2010(E), ``Systems and software engineering Vocabulary,'' First Edition, December 15, 2010.
  \item LWP 13620, ``Managing Information Technology Assets''
  \item SDD-513, `` RAVEN Software Design Description ''
  \item PLN-5552, `` RAVEN and RAVEN Plug-ins Software Quality Assurance and Maintenance and Operations Plan ''
\end{itemize}


\section{Definitions and Acronyms}

\subsection{Definitions}
\begin{itemize}
  \item \textbf{Baseline.} A specification or product (e.g., project plan, maintenance and operations [M\&O] plan, requirements, or
design) that has been formally reviewed and agreed upon, that thereafter serves as the basis for use and further
development, and that can be changed only by using an approved change control process. [ASME NQA-1-2008 with the
NQA-1a-2009 addenda edited]
  \item \textbf{Validation.} Confirmation, through the provision of objective evidence (e.g., acceptance test), that the requirements
for a specific intended use or application have been fulfilled. [ISO/IEC/IEEE 24765:2010(E) edited]
  \item \textbf{Verification.}
  \begin{itemize}
     \item The process of evaluating a system or component to determine whether the products of a given development
     phase satisfy the conditions imposed at the start of that phase.
     \item  Formal proof of program correctness (e.g., requirements, design, implementation reviews, system tests).
     [ISO/IEC/IEEE 24765:2010(E) edited]
  \end{itemize}
\end{itemize}

\subsection{Acronyms}
\begin{description}
\item[API] Application Programming Interfaces
\item[ASME] American Society of Mechanical Engineers
\item[DOE] Department of Energy
\item[HDF5] Hierarchical Data Format (5)
\item[LWRS] Light Water Reactor Sustainability
\item[NEAMS] Nuclear Energy Advanced Modeling and Simulation
\item[INL] Idaho National Laboratory
\item[IT] Information Technology
\item[M\&O] Maintenance and Operations
\item[NQA] Nuclear Quality Assurance
\item[POSIX]  Portable Operating System Interface
\item[PP]  Post-Processor
\item[QA] Quality Assurance
\item[RAVEN] Risk Analysis and Virtual ENviroment
\item[ROM] Reduced Order Model
\item[SDD] System Design Description
\item[XML] eXtensible Markup Language
\end{description}


\section{Pre-test Instructions/Environment/Setup}
The test of the requirements are performed automatically through
the CIS (Continuous Integration System) for each CR (Change Request).
The tests are performed on each supported Operative System (see \cite{RAVENuserManual}).

\input{traceability_matrix.tex}

 

    % ---------------------------------------------------------------------- %
    % References
    %

\addcontentsline{toc}{section}{Referenced Documents}
\bibliographystyle{ieeetr}
\bibliography{sr2ml_requirements_traceability_matrix}

\section*{Document Version Information}

\input{../../version.tex}

\end{document}
